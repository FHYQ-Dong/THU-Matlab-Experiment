\documentclass[utf8]{article}

\usepackage{ctex}
\usepackage{amsmath}
\usepackage{circuitikz}
\usepackage{tikz}
\usepackage{graphicx}
\usepackage{hyperref}
\usepackage{geometry}
\usepackage{pdfpages}
\usepackage{booktabs}
\usepackage{subfigure}
\usepackage{float}
\usepackage{amsmath}
\usepackage{amssymb}
\usepackage{mathrsfs}
\usepackage{multirow}
\usepackage{inputenc}
\usepackage{fancyhdr}
\usepackage{listings}
\usepackage[dvipsnames]{xcolor}
\usepackage{fontspec}
\usepackage{threeparttable}


\newfontfamily\codefont[
    Path = fonts/SarasaMonoSC/,
    Extension = .ttf,
    UprightFont = *-Regular,
    BoldFont = *-Bold,
    ItalicFont = *-Italic,
    BoldItalicFont = *-BoldItalic,
    Scale = MatchLowercase,
]{SarasaMonoSC}
\hypersetup{
    colorlinks=true,
    linkcolor=black,
    filecolor=black,      
    urlcolor=black,
    citecolor=black,
}
\lstdefinestyle{code}{ 
    basicstyle   = \codefont,
    breaklines   = true,
    breakindent  = 0pt,
    tabsize      = 4,
    keywordstyle = \bfseries\color{NavyBlue},
    emphstyle    = \bfseries\color{Rhodamine},
    commentstyle = \itshape\color{PineGreen},
    stringstyle  = \bfseries\color{Orange!90!black},
    columns      = fixed,
    numbers      = left,
    numbersep    = 2em,
    numberstyle  = \footnotesize,
    frame        = single,
    framesep     = 1em,
    keepspaces   = true,
    showspaces   = false,
    showstringspaces = false,
}
\lstset{style=code}


\geometry{a4paper, scale=0.8}
\pagestyle{fancy}
\fancyhf{}
\lhead{MATLAB音乐合成实验报告}
\rhead{FHYQ-Dong}
\cfoot{---~~\thepage~~---}
\setlength{\columnsep}{20pt}



\title{MATLAB音乐合成实验报告}
\author{FHYQ-Dong} 
\date{\zhtoday}

\begin{document}

\maketitle
\thispagestyle{fancy}

\section{简单的合成音乐}
\begin{enumerate}
    \item \label{ex:1}\textbf{要求:}请根据《东方红》片断的简谱和``十二平均律''计算出该片断中各个乐音的频率,在MATLAB中生成幅度为1、抽样频率为8kHz的正弦信号表示这些乐音。请用sound函数播放每个乐音,听一听音调是否正确。最后用这一系列乐音信号拼出《东方红》片断,注意控制每个乐音持续的时间要符合节拍,用sound播放你合成的音乐,听起来感觉如何? \\
        \textbf{实现:}事先根据十二平均律计算出各个音符的频率,存储至文件attachments/note2freq.mat中以备使用。通过产生简单的正弦信号即可实现音乐合成。代码见Listing \ref{code:ex-1}. \\
        \textbf{结果:}见文件attachments/ex\_1.wav.
    \item \label{ex:2}\textbf{要求:}你一定注意到实验 \ref{ex:1} 的乐曲中相邻乐音之间有``啪''的杂声,这是由于相位不连续产生了高频分量。这种噪声严重影响合成音乐的质量,丧失真实感。为了消除它,我们可以用包络修正每个乐音,以保证在乐音的邻接处信号幅度为零。此外建议用指数衰减的包络来表示。 \\
        \textbf{实现:}在实验 \ref{ex:1} 的基础上,增加了包络函数的实现。包络函数图像如图 \ref{fig:ex-2},代码见Listing \ref{code:ex-2}. \\
        \textbf{结果:}见文件attachments/ex\_2.wav.
        \begin{figure}[H]
            \centering
            \includegraphics[width=0.8\textwidth]{images/ex_2.png}
            \caption{实验 \ref{ex:2} 的包络函数}
            \label{fig:ex-2}
        \end{figure}
    \item \label{ex:3}\textbf{要求:}请用最简单的方法将实验 \ref{ex:2} 中的音乐分别升高和降低一个八度。(提示:音乐播放的时间可以变化)再难一些,请用resample函数(也可以用interp和decimate函数)将上述音乐升高半个音阶。(提示:视计算复杂度,不必特别精确) \\
        \textbf{实现:}在实验 \ref{ex:2} 的基础上,直接隔点抽取达到升高八度的效果;使用repelem函数实现降低八度的效果;使用resample函数实现升高半个音阶的效果。代码见Listing \ref{code:ex-3}. \\
        \textbf{结果:}见文件attachments/ex\_3\_u8.wav, attachments/ex\_3\_d8.wav, attachments/ex\_3\_u05.wav.
    \item \label{ex:4}\textbf{要求:}试着在实验 \ref{ex:2} 的音乐中增加一些谐波分量,听一听音乐是否更有```厚度''了?注意谐波分量的能量要小,否则掩盖住基音反而听不清音调了。(如果选择基波幅度为1,二次谐波幅度0.2,三次谐波幅度0.3,听起来像不像象风琴?) \\
        \textbf{实现:}在实验 \ref{ex:2} 的基础上,通过生成多个正弦信号并叠加的方式实现谐波分量的添加。实验中取一次、二次、三次谐波分量的幅度分别为1、0.2、0.3。代码见Listing \ref{code:ex-4}. \\
        \textbf{结果:}见文件attachments/ex\_4.wav.
    \item \label{ex:5}\textbf{要求:}自选其它音乐合成,例如贝多芬第五交响乐的开头两小节。 \\
        \textbf{实现:}合成了贝多芬第五交响乐的开头两小节。特别地,实现了和弦的合成。代码见Listing \ref{code:ex-5}. \\
        \textbf{结果:}见文件attachments/ex\_5.wav.
\end{enumerate}


\section{用傅里叶级数分析音乐}
\begin{enumerate}
    \setcounter{enumi}{5}
    \item \label{ex:6}\textbf{要求:}先用wavread函数载入attachment/fmt.wav文件,播放出来听听效果如何?是否比刚才的合成音乐真实多了? \\
        \textbf{实现:}使用wavread函数载入音频文件,并使用sound函数播放。代码见Listing \ref{code:ex-6}. \\
        \textbf{结果:}确实真实多了。
    \item \label{ex:7}\textbf{要求:}你知道待处理的wave2proc是如何从真实值realwave中得到的么?这个预处理过程可以去除真实乐曲中的非线性谐波和噪声,对于正确分析音调是非常重要的。提示:从时域做,可以继续使用resample函数。 \\
        \textbf{实现:}我猜想可以通过去除信号的直流分量来滤除噪声;通过一个带通滤波器来滤除非线性谐波,但不是非常确定,实现的结果看上去也不是非常像。代码见Listing \ref{code:ex-7}. \\
        \textbf{结果:}见图 \ref{fig:ex-7},我处理后的信号与wave2proc相比,整体上有缓慢起伏。怀疑是带通滤波器的影响。
        \begin{figure}[H]
            \centering
            \includegraphics[width=0.5\textwidth]{images/ex_7.png}
            \vspace{-2em}
            \caption{实验 \ref{ex:7} 的处理结果}
            \label{fig:ex-7}
        \end{figure}
    \item \label{ex:8}\textbf{要求:}这段音乐的基频是多少?是哪个音调?请用傅里叶级数或者变换的方法分析它的谐波分量分别是什么。提示:简单的方法是近似取出一个周期求傅里叶级数但这样明显不准确,因为你应该已经发现基音周期不是整数(这里不允许使用resample函数)。复杂些的方法是对整个信号求傅里叶变换(回忆周期性信号的傅里叶变换),但你可能发现无论你如何提高频域的分辨率,也得不到精确的包络(应该近似于冲激函数而不是 sinc 函数),可选的方法是增加时域的数据量,即再把时域信号重复若干次,看看这样是否效果好多了?请解释之。 \\
        \textbf{实现:}将原始信号重复10次后做FFT。这样效果变好的原因是:有限长信号可以看作是无限长信号乘上一个矩形窗。FFT后相当于无限长信号的频谱卷上sinc函数。在时域将信号重复多次相当于拉宽的矩形窗的宽度,从而频域sinc函数变窄,分辨率提高。代码见Listing \ref{code:ex-8}. \\
        \textbf{结果:}频谱图见图 \ref{fig:ex-8},基频:222.0459 Hz,音调:A3.
        \begin{figure}[H]
            \centering
            \includegraphics[width=0.5\textwidth]{images/ex_8.png}
            \caption{实验 \ref{ex:8} 的频谱图}
            \label{fig:ex-8}
        \end{figure}
    \item \label{ex:9}\textbf{要求:}再次载入attachments/fmt.wav,现在要求你写一段程序,自动分析出这段乐曲的音调和节拍! \\
        \textbf{实现:}使用FFT分析音调:取频谱的最大值对应的频率作为基音音调;使用自相关函数分析节拍:先求音频信号的能量(平方),再进行自相关。排除0延迟附近的值,找到自相关函数的第一个峰值对应的延迟作为节拍。代码见Listing \ref{code:ex-9}. \\
        \textbf{体会:}一开始分析节拍的时候并没有求信号的能量。这样求出来的BPM十分小(大约三十几)。在后续通过吉他频谱合成东方红音乐时发现三十几的BPM造成一拍内的音符数量过多,无法准确分析基音。 \\
        \textbf{结果:}音调:A3,222.05 Hz;节拍周期:0.50 秒,119.94 BPM.
\end{enumerate}


\section{基于傅里叶级数的合成音乐}
\begin{enumerate}
    \setcounter{enumi}{9}
    \item \label{ex:10}\textbf{要求:}用实验 \ref{ex:7} 计算出来的傅里叶级数再次完成实验 \ref{ex:4},听一听是否像演奏fmt.wav的吉他演奏出来的? \\
        \textbf{实现:}与实验 \ref{ex:11} 一起完成。
    \item \label{ex:11}\textbf{要求:}也许实验 \ref{ex:9} 还不是很像,因为对于一把泛音丰富的吉他而言,不可能每个音调对应的泛音数量和幅度都相同。但是通过完成实验 \ref{ex:8},你已经提取出fmt.wav中的很多音调,或者说,掌握了每个音调对应的傅里叶级数,大致了解了这把吉他的特征。现在就来演奏一曲《东方红》吧。提示:如果还是音调信息不够,那就利用相邻音调的信息近似好了,毕竟可以假设吉他的频响是连续变化的。 \\
        \textbf{实现:}合成流程如下:
        \begin{enumerate}
            \item 载入音频文件
            \item 分析音频的节拍信息
            \item 对每一拍,分析其基波和各次谐波
            \item 通过将《东方红》的音符临近匹配到存在数据的基音上,合成每个音符
            \item 将各个音符施加包络,合并成完整的旋律
        \end{enumerate}
        我一开始使用实验 \ref{ex:8} 的方法获取每拍内的基音,但是写到一半(listing \ref{code:music-synthesis-old})检查发现如此分析的结果是各个节拍内的基音频率都是一个值。后来询问ChatGPT\footnote{相关交流记录见此链接:\href{https://chatgpt.com/share/688b87a3-ab10-800e-a8f9-f033995d00f4}{https://chatgpt.com/share/688b87a3-ab10-800e-a8f9-f033995d00f4}}得知更加稳定的分析频谱的方式是:
        \begin{enumerate}
            \item 对每一拍内的原始音频信号进行预处理,包括去除直流分量、加窗等
            \item 通过自相关函数而非直接FFT分析基音频率
            \item 对每一拍内的音频信号进行时间整形:取整数个周期(采样率除以基音频率为一个周期内的采样点数)的音频信号做多周期平均、归一化等,得到单周期内的音频信号
            \item 对单周期内的音频信号做FFT,获取各次谐波的傅里叶级数
        \end{enumerate}
        其中,使用自相关函数而非FFT分析基音频率的原因有以下几点:
        \begin{enumerate}
            \item FFT若想达到较高的频域精度,时域的窗需要足够长。这样会跨越多个音符导致识别不准
            \item 谱峰往往出现在谐波,基波峰可能被掩盖
            \item 直接FFT可能出现一拍内有非整数个周期,造成频谱泄露混叠
        \end{enumerate}
        经过ChatGPT提示,最终完整的代码见Listing \ref{code:music-synthesis}. \\
        \textbf{结果:}合成的音乐见文件attachmentss/dongfanghong\_synth.wav
\end{enumerate}

\newpage
\appendix
\section{源代码}

\subsection{项目文件结构}
\label{subsec:project-structure}
项目文件结构如下。运行MATLAB程序时,请确保当前工作目录为项目根目录。以下代码与压缩包内的MATLAB源文件完全一致。
\begin{lstlisting}[language={bash}, numbers=none]
    /
    ├── attachments/ # 存放附件
    │   ├── fmt.wav                # 音频文件
    │   ├── 音乐合成大作业.pdf     # 音乐合成大作业的指导书
    │   ├── Guitar.MAT             # 吉他音色的MATLAB数据文件
    │   ├── note2freq.mat          # 音符到频率的映射表
    │   ├── ex_1.wav               # 实验1产生的音频文件
    │   ├── ex_2.wav               # 实验2产生的音频文件
    │   ├── ex_3_u8.wav            # 实验3产生的音频文件(升8度)
    │   ├── ex_3_d8.wav            # 实验3产生的音频文件(降8度)
    │   ├── ex_3_u05.wav           # 实验3产生的音频文件(升0.5度)
    │   ├── ex_4.wav               # 实验4产生的音频文件
    │   ├── ex_5.wav               # 实验5产生的音频文件
    │   ├── ex_7.png               # 实验7产生的波形图
    │   ├── ex_8.png               # 实验8产生的波形图
    │   └── dongfanghong_synth.wav # 合成的东方红音频文件
    ├── ex_1.m                     # 实验1的代码
    ├── ex_2.m                     # 实验2的代码
    ├── ex_3.m                     # 实验3的代码
    ├── ex_4.m                     # 实验4的代码
    ├── ex_5.m                     # 实验5的代码
    ├── ex_6.m                     # 实验6的代码
    ├── ex_7.m                     # 实验7的代码
    ├── ex_8.m                     # 实验8的代码
    ├── ex_9.m                     # 实验9的代码
    ├── music_synthesis.m          # 音乐合成的主程序
    └── music_synthesis_old.m      # 尝试失败的音乐合成程序
\end{lstlisting}

\subsection{MATLAB源代码}
\lstinputlisting[language={MATLAB}, caption={ex\_1.m}, label={code:ex-1}]{../ex_1.m}
\lstinputlisting[language={MATLAB}, caption={ex\_2.m}, label={code:ex-2}]{../ex_2.m}
\lstinputlisting[language={MATLAB}, caption={ex\_3.m}, label={code:ex-3}]{../ex_3.m}
\lstinputlisting[language={MATLAB}, caption={ex\_4.m}, label={code:ex-4}]{../ex_4.m}
\lstinputlisting[language={MATLAB}, caption={ex\_5.m}, label={code:ex-5}]{../ex_5.m}
\lstinputlisting[language={MATLAB}, caption={ex\_6.m}, label={code:ex-6}]{../ex_6.m}
\lstinputlisting[language={MATLAB}, caption={ex\_7.m}, label={code:ex-7}]{../ex_7.m}
\lstinputlisting[language={MATLAB}, caption={ex\_8.m}, label={code:ex-8}]{../ex_8.m}
\lstinputlisting[language={MATLAB}, caption={ex\_9.m}, label={code:ex-9}]{../ex_9.m}
\lstinputlisting[language={MATLAB}, caption={music\_synthesis\_old.m}, label={code:music-synthesis-old}]{../music_synthesis_old.m}
\lstinputlisting[language={MATLAB}, caption={music\_synthesis.m}, label={code:music-synthesis}]{../music_synthesis.m}


\end{document}
